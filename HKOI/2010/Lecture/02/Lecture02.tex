%% LyX 1.6.5 created this file.  For more info, see http://www.lyx.org/.
%% Do not edit unless you really know what you are doing.
\documentclass[size=8pt,english,style=fyma,display=slidesnotes,mode=present,paper=screen]{powerdot}
\usepackage[T1]{fontenc}
\usepackage[utf8]{inputenc}
\usepackage{setspace}
\usepackage{amsthm}
\usepackage{amsmath}
\usepackage{tikz}
\usepackage{graphics}
\onehalfspacing

\makeatletter

%%%%%%%%%%%%%%%%%%%%%%%%%%%%%% LyX specific LaTeX commands.
\pdfpageheight\paperheight
\pdfpagewidth\paperwidth

	%%%%%%%%%%%%%%%%%%%%%%%%%%%%%% Textclass specific LaTeX commands.
	\theoremstyle{plain}
	\newtheorem*{thm*}{Theorem}
	\newtheorem{thm}{Theorem}	
	\newtheorem*{def*}{Definition}
\makeatother

\usepackage{babel}

\begin{document}
	
\title{HKOI Training}
\author{$ami\sim wkc$}
\date{Last modified: \today}
\maketitle
\lyxend\section[toc=Lecture 02]{Lecture 02 \\ Propositional Logic and Grammar in C}

\section{Propositional Logic}
\begin{slide}{Proposition / Statement}
	A \emph{statement} / \emph{proposition} is a sentence that has either an answer, ``Yes'' or ``No''.\footnote{We skip a bit by using ``common sense'' to determine whether a sentence is a proposition or not.}\pause
	
	For example, all the following are proposition.\pause
	\begin{itemize}
		\item	Today is hot.\pause
		\item	I will not go to school.\pause		
		\item	$1+2+3=\frac{1}{2}\left(3\right)\left(4\right)$. \pause (Yes) \pause
		\item	There are infinitely many prime numbers. \pause (Yes) \pause
		\item	$\left(\sqrt{x^2}\right)^2 = x$. \pause (No) \pause
		\item	If $n$ is a 5-digit square integer, then $n=29929$. \pause (Yes)
		\item	$x=2$ only if $x^2=4$. \pause	(Yes)
		\item	$x=2$ if $x^2=4$. \pause	(No)
		\item	$n=2$ and $n$ is a prime. \pause	(Yes)
	\end{itemize}
\end{slide}
\begin{slide}[toc=]{Proposition / Statement}
	The following are not propositions or we won't discuss the following kind of sentences.

	\begin{itemize}
		\item	What time is it now?\pause
		\item	\pause (empty string)\footnote{This is usually called the $\epsilon$-string} \pause		
		\item	This statement is false. \pause
		\item	I am lying. \footnote{The Liar paradox}\pause
		\item	The second unique child of God is a female. \pause
	\end{itemize}
	
	Actually, some of them can be considered as statements. 
	
	However, for simplicity, we shall avoid them at this moment.
\end{slide}
\begin{slide}{Proposition operators}
	Given some propositions, we can construct another proposition from them.
	
	\begin{itemize}
		\item	What time is it now?\pause
		\item	\pause (empty sentence) \pause		
		\item	This statement is false. \pause
		\item	I am lying. \pause
		\item	The second unique child of God is a female. \pause
	\end{itemize}
\end{slide}

\section{End}
\end{document}
