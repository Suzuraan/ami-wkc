%% LyX 1.6.5 created this file.  For more info, see http://www.lyx.org/.
%% Do not edit unless you really know what you are doing.
\documentclass[size=8pt,english,style=fyma,display=slidesnotes,mode=present,paper=screen]{powerdot}
\usepackage[T1]{fontenc}
\usepackage[utf8]{inputenc}
\usepackage{setspace}
\usepackage{amsthm}
\usepackage{amsmath}
\usepackage{tikz}
\usepackage{graphics}
\usepackage{polynom}
\onehalfspacing

\makeatletter


%%%%%%%%%%%%%%%%%%%%%%%%%%%%%% LyX specific LaTeX commands.
\pdfpageheight\paperheight
\pdfpagewidth\paperwidth

	%%%%%%%%%%%%%%%%%%%%%%%%%%%%%% Textclass specific LaTeX commands.
	\theoremstyle{plain}
	\newtheorem*{thm*}{Theorem}
	\newtheorem*{def*}{Definition}
\makeatother

\usepackage{babel}

\begin{document}
	
\title{HKOI Training}
\author{$ami\sim wkc$}
\date{Last modified: \today}
\maketitle
\lyxend\section[toc=Lecture 03]{Lecture 03 \\ Mathematical reasoning.\\ Combinatorics and algorithms.\\ Variables in C}

\section{Mathematical reasoning}
\begin{slide}{Verification of propositions}
	In last lecture, we discussed about the construction of propositions. \pause
	
	We also talked about the association of the truth values of those further constructed propositions. \pause
	
	All in all, we still need to know the truth value of the very first proposition. \pause
	
	For example,\pause
	
	\begin{itemize}
		\item	$2$ is a prime number.\pause
		\item	Pythagoras theorem is true\footnote{Read the solution of week 1 exercise for complete proof}.\pause		
		\item	$1+2+3=\frac{1}{2}\left(3\right)\left(4\right)$. \pause (Yes) \pause
		\item	There are infinitely many prime numbers. \pause (Yes) \pause
		\item	$\sqrt{x^2} = x$. \pause (No, it is false when x is negative.) \pause
		\item	If $n$ is a 5-digit square integer, then $n=29929$. \pause (No)
		\item	$x=2$ only if $x^2=4$. \pause	(Yes)
		\item	$x=2$ if $x^2=4$. \pause	(No)
		\item	$n=2$ and $n$ is a prime. \pause	(Yes)
	\end{itemize}
\end{slide}

\begin{slide}[toc=]{Proposition / Statement}
	The following are not propositions or we won't discuss the following kind of sentences. \pause
	
	\begin{itemize}
		\item	What time is it now?\pause
		\item	\pause (empty string)\footnote{This is usually called the $\epsilon$-string} \pause		
		\item	This statement is false. \pause
		\item	I am lying. \footnote{The Liar paradox}\pause
		\item	The second unique child of God is a female. \pause
	\end{itemize}
	
	Actually, some of them can be considered as statements. 
	
	However, for simplicity, we shall avoid them at this moment.
\end{slide}
\begin{slide}{Proposition operators}
	Given some propositions, 
	
	we can create new propositions from them by using \emph{logical connectives}. \pause
	
	Be careful, we don't interpret the meaning at this stage. \pause
	
	For example\footnote{We don't care about grammar or tense. What we are interested in the new proposition only.}, \pause
		
	\begin{itemize}
		\item	NOT(Today is hot).\pause
		\item	NOT(I will not go to school). \pause		
		\item	Today is hot AND I will go to school. \pause
		\item	If today is hot, then I will not go to school. \pause
		\item	$x>3$ OR $x<-1$. \pause		
		\item	Every $x$ is greater than $3$. \pause
		\item	There is a number which is less than $-1$ or greater than $3$. \pause		
	\end{itemize}
\end{slide}
\begin{slide}{Negation - NOT}
	The negation of a proposition $P$ is $\sim{}P$. \pause
	
	Some book use $\lnot P$ to denote the negation. \pause
	
	It is simply a proposition prefixed by a word ``not''.
	
	\begin{itemize}
		\item	NOT(Today is hot).\pause
		\item	NOT(I will not go to school). \pause		
		\item	NOT($x>3$). \pause
		\item	NOT($x$ is a prime). \pause		
	\end{itemize}
\end{slide}
\begin{slide}{Conjunction - AND}
	The conjunction of two propositions $P,Q$ is $(P)\land{}(Q)$. \pause
	
	We will denote the conjunction usually by $(P)\, and\, (Q)$ instead. \pause
	
	It connects two propositions by adding by a word ``and''.\pause
	
	We may sometimes omit the parentheses as well as long as the meaning is clear.\pause
	
	\begin{itemize}
		\item	(Today is hot) AND (I will go to school).\pause
		\item	Today is hot AND I will go to school.\pause		
		\item	NOT(I will not go to school) AND NOT(Today is hot). \pause
		\item   ($x>2$) AND ($x$ is even). \pause		
	\end{itemize}
\end{slide}

\section{Grammar in C}
\begin{slide}{English Grammar}
	For those who study linguistic, they view the grammar of English systematically.

	\begin{itemize}
		\item 	Each passage consists of a title, author and a sequence of paragraphs  \pause 
		\item 	Each paragraph is a sequence of sentences  \pause 		
		\item	Every sentence end with a full-stop (.) or an exclamation mark (!). \pause
		\item	\mbox{<sentence> $::=$ <subject> <verb> [<object>]} \footnote{Every sentence consists of a subject and a verb, object is optional} \pause
		\item	Every subject is a noun-phrase, verb-phrase etc. \pause
		\item 	Every noun-pharse are consists of a smaller noun-phrase, noun, relative pronoun etc. \pause
		\item 	Every noun are vocabulary. \pause
		\item	Every vocabulary is consisting of correctly spelled character sequences. \pause
		\item	A character sequence is consists of characters \pause
		\item 	Each character is from 'a' to 'z' or 'A' to 'Z' \pause
	\end{itemize}
\end{slide}

\begin{slide}{C Grammar - BNF Grammar}
	An analogue linguistic structure for programming language C also exists.
	
	\begin{itemize}
		\item	Each program consists of pre-processor directives and functions\footnote{A part to tell the computer to store action.} \pause
		\item	Each function consists of statements. \pause
		\item	Every statements ends with a semi-colon (;). \pause
		\item	Every statement is either a control flow , variable declaration or expression. \pause
		\item	expression can be arithmetic expression, logical expression etc. \pause
		\item	arithmetic expression consists consists of addition, subtraction etc. \pause
		\item	\mbox{<addition>$::=$<Arithmetic Operand 1> '+' <Arithmetic 2>} \pause
		\item	Arithmetic Operand can be function values, variables names or numbers. \pause
		\item	numbers can be integers or real numbers. \pause
		\item	\mbox{<integers>$::=$ [0-9]+}, i.e. at least one of any one of '0','1', $\cdots$, '9' \pause
	\end{itemize}
\end{slide}

\begin{slide}{Arithmetic Expression}
	We learnt the way to convert usual mathematical expression into C language. \pause
	
	However, how could we tell the computer to do the following two different action? \pause
	
	\medskip{}
	
	\twocolumn[]{
		$3\div{}2$ 
		
		$3\div{}2$		
	}{
		Quotient: 1  \pause
		
		Real division: 1.5 \pause
	}
	
	\pause

	Even ourselves cannot distinguish the two different division without further explanation.\pause
	
	Computer will use the following rules to distinguish the two different division.\pause
	
	\medskip{}
	
	\begin{enumerate}
		\item	If every operands are computer-integers, it perform the quotient division.\footnote{Computer like doing discrete mathematics too!} \pause
		\item	If any one of the operand is not a computer-integer, it will switch to real division. \pause
	\end{enumerate}
	
	\medskip{}

	Computer-integers means that the number is ``written in a form''
	
	so that the computer treat it as an integer. \pause
	
	For example, we can say $2.3$ is not a whole number due to the decimal place.
	
\end{slide}

\begin{slide}{Regular Expression}
	To specify clear what we mean by ``written in a form'', we need \textbf{Regular Expression}. \pause

	It is a tool used for string matching. \pause
	
	For example, \pause
	
	\begin{itemize}
		\item[\mbox{[0-9]}]  matches any one of the character '0','1',$\cdots$,'9' \pause
		\item[\mbox{[0-9]+}] matches any string that is consists of at least one digit. \pause
		\item[\mbox{a*bc}] matches any string that is end with ``bc'' and start with any number of ``a'. \pause
		\item[\mbox{[\textasciicircum{}a-z]}] matches any character that is not one of 'a','b',$\cdots$,'z'. \pause
		\item[\mbox{ak4|bb10}] matches any one of the string ``ak4'' or ``bb10''. \pause
		\item[\mbox{(ab)|(kaab)}] matches any one of the string ``ab'' or ``kaab''\footnote{The parentheses here is used for grouping}. \pause
		\item[\mbox{(a|b)+}] matches any string that is consists of 'a' and 'b' and is non-empty. \pause
		\item[\mbox{a|b+}] matches any string ``a'' , ``b'' , ``bb'', ``bbb'' , $\cdots$. \pause
		\item[\mbox{(\textbackslash{}+|\textbackslash{}-)?}] matches any string ``+'' , ``-'' or $\epsilon$-string , $\cdots$. \pause		
	\end{itemize}
\end{slide}

\begin{slide}{Computer Integers and Real numbers}
	Although we won't distinguish usually between $2.5$ and $3$, there are just number. \pause
	
	Computer would like to do so. \pause
	
	It only knows whole number. \pause
	
	\medskip{}
	
	Computer integer is of the form (\textbackslash{}+|\textbackslash{}-)?[0-9]+ \pause
	
	Therefore, it will treat $3/2$ to be $1$. \pause
	
	\medskip{}
	
	For simplicity, we denote [DIGIT] to be [0-9] and [INT] to be [DIGIT]+ \pause
	
	Computer treats other numbers to be computer-real number, it can be ``-2.3'' , ``2.0'', ``1E+12''\footnote{Scientific notation $1\times{}10^{12}$.}. \pause
	
	Let [SIGN] $::=$ (\textbackslash{}+|\textbackslash{}-), \pause
	
	Computer real number is of the form [SIGN]?[DIGIT]*''.''[DIGIT]+((e|E)[SIGN]?[INT])  or [SIGN]?[DIGIT]+(e|E)[SIGN]?[INT]. \pause
	
	\medskip{}	
	
	Therefore, it will treat $3/2.0$ to be $1.5$ , $3.0/2$ to be $1.5$ and $3.0/2.0$ to be $1.5$. \pause
\end{slide}
\section{End}
\end{document}
