%% LyX 1.6.5 created this file.  For more info, see http://www.lyx.org/.
%% Do not edit unless you really know what you are doing.
\documentclass[size=8pt,english,style=fyma,display=slidesnotes,mode=present,paper=screen]{powerdot}
\usepackage[T1]{fontenc}
\usepackage[utf8]{inputenc}
\usepackage{setspace}
\usepackage{amsthm}
\usepackage{amsmath}
\usepackage{tikz}
\usepackage{graphics}
\usepackage{polynom}
\usepackage{algorithm2e}
\onehalfspacing

\makeatletter


%%%%%%%%%%%%%%%%%%%%%%%%%%%%%% LyX specific LaTeX commands.
\pdfpageheight\paperheight
\pdfpagewidth\paperwidth

	%%%%%%%%%%%%%%%%%%%%%%%%%%%%%% Textclass specific LaTeX commands.
	\theoremstyle{plain}
	\newtheorem*{thm*}{Theorem}
	\newtheorem*{def*}{Definition}
	\newtheorem*{axiom*}{Axiom}
\makeatother

\usepackage{babel}

\begin{document}
	
\title{HKOI Training}
\author{$ami\sim wkc$}
\date{Last modified: \today}
\maketitle
\lyxend\section[toc=Lecture 03]{Lecture 03 \\ Mathematical reasoning.\\ Control flows and algorithms.\\ Variables in C}

\section{Mathematical reasoning}
\begin{slide}{Verification of propositions}
	In last lecture, we discussed about the construction of propositions. \pause
	
	We know the association of the truth values of those further constructed propositions. \pause
	
	However, we did not mention how one could know the truth value of a given proposition. \pause
	
	For example,\pause
	
	\begin{itemize}
		\item	$2$ is a prime number.\pause
		\item	Pythagoras theorem is true.\footnote{Read the solution of week 1 exercise for complete proof.}\pause		
		\item	Angle sum of triangle is always $180^\circ$. \pause (What is the meaning of ``always'' here?)\pause
		\item	The Extended Euclidean algorithm is correct.\footnote{A proposition involving recursive definition. It will be proved in number theory.} \pause (Implicitly ``always'') \pause				
		\item	Today is hot. \pause (Another way to verify)\pause
		\item	$\forall x,\sqrt{x^2} = x$. \pause (Counterexample is demonstrated.) \pause
		\item	If $n$ is a 5-digit square integer, then $n=29929$.\footnote{What is its truth value, when $n$ is $10$, $29929$, $10000$?} \pause (Mostly yes.) \pause
	\end{itemize}
\end{slide}

\begin{slide}[toc=Ambiguity of IF-THEN]{Ambiguity of conditional proposition}
	Let's discuss a conditional proposition with a propositional variable,
	
	$P\left(n\right):=$``If $n$ is a 5-digit square integer, then $n=29929$''.
	
	\pause
	
	Indeed, ``If $n$ is a 5-digit square integer, then $n=29929$'' is not a well-defined proposition.
	\footnote{You cannot tell whether it is true or not by merely reading the $n$ as a variable} \pause
	
	To discuss the truth value, one need to give a value to $n$. \pause	
	
	For example, \pause
	
	\begin{itemize}
		\item	$P\left(2\right)$ is true as $2$ is not a five digit integer.\pause
		\item	$P\left(10000\right)$ is false. \pause
		\item	$P\left(29929\right)$ is true. \pause
	\end{itemize}
	
	In mathematics, we would usually interpret the sentence ''$Q\left(n\right)\implies P\left(n\right)$'' to be
	
	``$\forall n$, $Q\left(n\right)\implies P\left(n\right)$''. \pause
	
	\medskip
	
	That's how we say $P\left(n\right)$ is false without explicitly giving a value to $n$.\pause
	
	Similarly, the word ``always'' has the implicit ``FOR-ALL'' meaning.
	
	
\end{slide}
\begin{slide}[toc=Direct Proof]{Direct Proof - Proof from definition : $n\cdot 0=0$}
	How to prove / disprove the proposition ``$\forall n,0\cdot n=0$''? \pause
	
	To verify this above proposition, we need to know about what do we mean by $0$. \pause
	
	\begin{def*}
		$0$ is the number such that for any $n$, $n+0=n$.
	\end{def*}

	\begin{def*}
		$-n$ is the number such that $n+ \left(-n\right)=0$.
	\end{def*}

	\pause	
	
	\begin{proof}
		We use the distributive law and the definition of $0$, \pause
		
		$n\cdot 0 = n \cdot \left(0+0\right) = n\cdot 0 + n\cdot 0$ \pause
		
		Adding $-\left(n\cdot 0\right)$ to both side, we get $n\cdot 0=0$.
	\end{proof}
	
	\pause
	
	This proof seems meaningless, however, remember what properties we used. \pause
	
	\medskip
	
	Later on, if there is another kind of arithmetic system also having the same properties, \pause
	
	we can claim that the above still holds.\pause
	
	For example, the modular arithmetic, i.e. doing arithmetic on a clock.
	
\end{slide}

\begin{slide}[toc=]{Direct Proof - Proof from definition : $\left(-1\right)\left(-1\right)=1$}
	How to prove / disprove the proposition ``$\left(-1\right)\left(-1\right)=1$''? \pause
	
	\begin{def*}
		$1$ is the number such that for any non-zero $n$, $1\cdot n=n$.
	\end{def*}
	
	\begin{proof}
		$\left(1+\left(-1\right)\right)^2 = 1^2 + 1 \times \left(-1\right) + \left(-1\right)\times 1 + \left(-1\right)\left(-1\right)$\pause
		
		Using the definition of $1$, we have, $1^2=1$ , $1 \times \left(-1\right)=-1$ and $ \left(-1\right)\times 1 = -1$.\pause
		
		From the definition of $-n$, we have  $\left(1+\left(-1\right)\right)=0$ and  $\left(1+\left(-1\right)\right)^2 = 0\cdot 0=0$.\pause

		Therefore, we have $0=1 + \left(-1\right) + \left(-1\right) + \left(-1\right)\left(-1\right)$.\pause
		
		Evaluating from left to right, $1 + \left(-1\right)$ becomes $0$. \pause
		
		This gives,  $0= \left(-1\right) + \left(-1\right)\left(-1\right)$. \pause
		
		Finally, adding $1$ to the left of both sides, we have $1 = 1 +  \left(-1\right) + \left(-1\right)\left(-1\right)$ \pause
		
		Therefore, $1 =  \left(-1\right)\left(-1\right)$.
		
	\end{proof}
	
	\pause
	
	We will come back to these proofs at Abstract Algebra.
\end{slide}
\begin{slide}[toc=Proof by counterexample]{Proof by example / counterexample}
	Usually, to deal with propositions with FOR-ALL, \pause
	
	one can prove its falsity by showing a counterexample.\pause
	
	\medskip
	
	``Every people eat meat'', this is false as we can find a vegetarian.\pause
	
	\medskip
	
	Similarly, one could show the correctness of a THERE-EXISTS statement \pause
	
	by giving a particular example.
	
	\medskip
	
	``There is a pair of integers $x$ and $y$ such that $7x+9y=1$''. \pause

	\medskip
	
	$7\left(-5\right)+9\left(4\right)=1$, hence the above proposition is true.
\end{slide}
\begin{slide}{Verifying everything}
	However, to prove the correctness of a FOR-ALL statement, \pause
	
	or the falsity of a THERE-EXISTS statement, \pause
	
	we need another way. \pause
	
	For example, the following statements cannot be proved easily in logic \pause
	
	\medskip
	
	\begin{itemize}
		\item	Every horses have four legs. \footnote{You need to grab all the horses and observe one by one} \pause
		\item 	There is no any extraterrestrial life. \footnote{You need to observe everywhere outside the Earth}\pause
	\end{itemize}
	
\end{slide}
\begin{slide}{Proof by induction}
	Given the \textbf{domain of discourse} is the positive integers. \pause
	
	One can use a tool called \textbf{Mathematical Induction} to prove a FOR-ALL statement. \pause
		
	\medskip
	
	Let $P(n)::=\mbox{''} 1+2+\cdots+n \mbox{ is always equal to } \frac{1}{2}n(n+1)\mbox{"}$. \pause 
	
	We can check $P(1),\dots ,P(100)$ one by one. \pause
	\begin{itemize}
		\item	$1 = \frac{1}{2}(1)(2)$  \pause
		\item 	$1+2 = 3 = \frac{1}{2}(2)(3)$ \pause
	\end{itemize}
	
	However, we still don't know whether $P(101)$ is true or not. \pause
	
	Even you can check the proposition up to $100000$, \pause
	
	there are still infinitely many propositions to be verified. \pause
	
	To verify all of them, we cannot check one by one.\footnote{You may use a direct proof to show}
\end{slide}
\begin{slide}{Well ordering principle}
	\begin{axiom*}[Well ordering principle]
		Let $\mathcal{S}$ be a collection of positive integers.
		
		If $\mathcal{S}$ is non-empty, then it has the smallest integer.
		
		i.e. Every non-empty collection of positive integers has the smallest integer.
	\end{axiom*}
	
	The above axiom says that
	
	you can always find the smallest integer whenever you have some integers.\pause
	
	\medskip
	
	For example, we can mention about \pause
	
	\medskip
	
		\begin{itemize}
			\item the smallest positive even number.\pause
			\item the smallest positive odd number.\pause
			\item the smallest common multiple of $15$ and $20$.\pause
			\item the smallest positive prime.\pause
		\end{itemize}
\end{slide}

\begin{slide}{Idea of induction}
	Assume the \textbf{domain of discourse} is the positive integers.
	
	\pause
	
	\begin{thm*}[Mathematical induction] Let $P(n)$ be a proposition with propositional variable $n$. \pause
		Suppose the following: \pause
			\begin{enumerate}
				\item	$P(1)$ is true.
				\item	$\forall k, P(k)\implies P(k+1)$ is true.
			\end{enumerate}
		\pause
		
		Then $\forall k,P(k)$ is true
	\end{thm*}
	
	\pause
	
	Therefore, to show that a FOR-ALL proposition,  \pause
	
	one can check $P(1)$ is true and prove the second conditional statement.
	
	\pause
	
	\medskip
	
	The first condition is called the \textbf{base case}.
	
	The second condition is called the \textbf{induction step}.
	
\end{slide}
\begin{slide}{Proof of induction}
	\begin{proof}
		Let $\mathcal{S}$ be the collection of positive integers $n$ such that $P(n)$ is false.\pause
		
		we want to show that this collection is empty.\pause
		
		Let's see what happen if it is non-empty.\pause
		
		Suppose it is non-empty, the well-ordering principle tells that the smallest integer exists. \pause
		
		i.e. There is a smallest integer $m$  such that $P(m)$ is false.\pause
		
		Denote that integer by $m$ and clearly $m\neq 1$ since $P(1)$ is true. \pause
		
		Therefore, $m>1$  and $m-1$ is again a positive integer. \pause
		
		Since $m$ is the smallest one such that $P(m)$ is false, we have $P(m-1)$ is true.\pause
		
		Finally, the second part of our assumption, tells that $P(m-1)\implies P(m)$ is true.\pause
		
		Since $P(m-1)$ is true , we must have  $P(m)$ be true. \pause (contradiction, as $P(m)$ is false.) \pause
		
		Therefore, the very most assumption of ``$\mathcal{S}$ is non-empty'' is false. \pause
		
		Hence, there is no any positive integer $n$ such that $P(n)$ is false. \pause
		
		``$\forall n,P(n)$'' is logically equivalent to ``$\sim \exists n, \sim P(n)$''\pause
		
		Therefore, $\forall n,P(n)$ is true.
	\end{proof}
	
	\pause
	
	The above argument works for any propositions as long as it satisfy our assumption. \pause
	
	Usually, after we checked the assumption holds, we can use this theorem to conclude. \pause
	
	By ``using a theorem'', we actually means that we re-apply the arguments of a proof again.
\end{slide}
\section{Induction example}
\begin{slide}{Sum of power one}
	Let $P(n)::=\mbox{''} 1+2+\cdots+n \mbox{ is always equal to } \frac{1}{2}n(n+1)\mbox{"}$. \pause 
	
	we shall show that ``$\forall n,P(n)$'' by using the mathematical induction. \pause
	
	\begin{proof}
	
	Firstly, we check that $P(1)$ is true, which is obvious in this case. \pause
	
	Secondly, for each given $k$, \pause
	
	we want to show that the proposition $P(k)\implies P(k+1)$ is also true. \pause

	To prove a conditional proposition always holds, \pause
	
	we need to check all the possible truth value associated between the two statements. \pause
	
	If $P(k)$ is false, then the whole conditional statement is true. \pause
	
	Remaining is to verify the case for $P(k)$ is true to show $P(k+1)$ is true. \pause
	
	i.e. $1+2+\cdots +k=\frac{1}{2}k(k+1)$  is true.
	
	Now, consider $1+2+\cdots +k + (k+1) = (1+2+\cdots +k) + k+1$. \pause
	
	Our assumption says that $(1+2+\cdots +k) + k+1=  \frac{1}{2}k(k+1) + (k+1)$ \pause 
	
	$\frac{1}{2}k(k+1) + (k+1) = \frac{1}{2}(k+1)((k+1)+2)$ \pause	
	
	Hence, $1+2+\cdots +k + (k+1) = \frac{1}{2}(k+1)((k+1)+2)$ , i.e. $P(k+1)$ is true. \pause

	By induction, we know that $\forall k,P(k)$ is true.
	
	\end{proof}
	
	\pause
	
	Try to show the above proposition using a direct proof.
\end{slide}

\begin{slide}{Sum of power two}
	Let $P(n)::=\mbox{''} 1^2+2^2+\cdots+n^2 \mbox{ is always equal to } \frac{1}{6}n(n+1)(2n+1)\mbox{"}$. \pause 
	
	\begin{proof}
		
		Firstly, we check that $P(1)$ is true, which is obvious in this case. \pause
		
		Secondly, for each given $k$, \pause
		
		assume $P(k)$ is true, i.e. $1^2+2^2+\cdots +k^2=\frac{1}{6}k(k+1)(2k+1)$. \pause
		
		Now, consider $1^2+2^2+\cdots +k^2 + (k+1)^2 = (1^2+2^2+\cdots +k^2) + (k+1)^2$. \pause
		
		Hence, $1^2+2^2+\cdots +k^2 + (k+1)^2 = \frac{1}{6}(k)(k+1)(2k+1) + (k+1)^2$ \pause
		
		$\frac{1}{6}(k)(k+1)(2k+1) + (k+1)^2 = \frac{1}{6}(k+1)(k(2k+1) + 6(k+1))$ \pause
		
		$k(2k+1) + 6(k+1) = 2k^2 + 7k + 6 = (2k+3)(k+2)$ \pause 
		
		Hence, $\frac{1}{6}(k+1)(k(2k+1) + 6(k+1))=\frac{1}{6}(k+1)(k+2)(2k+3)$.\pause
		
		$\frac{1}{6}(k+1)(k+2)(2k+3) = \frac{1}{6}(k+1)((k+1)+1)(2(k+1)+1)$ \pause
		
		Hence, $P(k+1)$ is true. \pause
		
		By induction, $\forall k,P(k)$ is true. \pause
		
	\end{proof}
	
	How can one come up with the expression at the right hand side?

\end{slide}

\begin{slide}{Summation notation}
	Let $f(i)$ be a function of $i$, e.g. $\sin$ , $i^2$. \pause
	
	and $n,m$ be two integers with $n<m$. \pause
	
	\medskip
	
	The summation notation $\sum\limits_{i=n}^{m} f(i)$ is defined as $f(n)+f(n+1)+\cdots +f(m)$. \footnote{This is just a notation to simplify our mathematical expression.}\pause
	
	\medskip
	
	Literally, it means keep adding from $i=n$ up to $i=m$. \pause
	
	\medskip
	
	For example, 
	
	\begin{itemize}
		\item $\sum\limits_{i=1}^{m} i$ is the same as $1 + 2 + \cdots + m$. \pause
		
		\item $\sum\limits_{i=3}^{m} i^2$ is the same as $3^2 + 4^2 + \cdots + m^2$. \pause
	\end{itemize}
	
	The previous two identities can be written as 

	\begin{itemize}
		\item $\sum\limits_{i=1}^{m} i=\frac{1}{2}m(m+1)$. \pause
		
		\item $\sum\limits_{i=1}^{m} i^2=\frac{1}{6}m(m+1)(2m+1)$. \pause
	\end{itemize}
\end{slide}

\begin{slide}{Properties of summation}
	Let $f,g$ be functions, $n,m$ be integers with $n<m$ and $c$ be a constant.
	
	\begin{thm*} The following are true:
		\begin{enumerate}
			\item If $n\leq k<m$, then $\sum\limits_{i=n}^{m} f(i) = \sum\limits_{i=n}^{k} f(i) + \sum\limits_{i=k+1}^{m} f(i)$. \pause
			\item $\sum\limits_{i=1}^{n} c = nc$ (adding total $n$'s $c$ is $nc$.) \pause
			\item $\sum\limits_{i=n}^{m} c = (m-n+1)c$\pause
			\item $\sum\limits_{i=n}^{m} c\cdot f(i) = c\cdot \sum\limits_{i=n}^{m} f(i)$\pause
			\item $\sum\limits_{i=n}^{m} \left[f(i)+g(i)\right] = \left(\sum\limits_{i=n}^{m} f(i)\right)+\left(\sum\limits_{i=n}^{m}g(i)\right)$\pause						
		\end{enumerate}
	\end{thm*}
	The first: ``adding from $n$ to $m$ is the same as adding from $n$ to $k$ then from $k+1$ to $m$''.\pause
	
	The forth: $\left(cf(n)+cf(n+1)+\cdots +cf(m)\right) = c\left(f(n)+f(n+1)+\cdots +f(m)\right)$\pause
\end{slide}
\begin{slide}[toc=]{Properties of summation}
	The fifth: $\left(f(n)+g(n)\right) + \left(f(n+1)+g(n+1)\right) +\cdots \left(f(m)+g(m)\right) = \left(f(n)+f(n+1)+\cdots+f(m)\right) + \left(g(n)+g(n+1)+\cdots +g(m)\right)$\pause
	
	\medskip
	
	i.e. we can rearrange the terms as long as there are finitely many.\pause
	
	\begin{proof}
		Let $P(n)$ be the proposition that ``$\sum\limits_{i=1}^{n} c = nc$''. \pause
		
		$P(1)$ is true, as adding from $m$ to $m$ means added once only.\pause
		
		Assume $\sum\limits_{i=1}^{k} c = kc$, then $\sum\limits_{i=1}^{k+1} c = \sum\limits_{i=1}^{k} c + \sum\limits_{i=k+1}^{k+1} c = kc + c = (k+1)c$\pause
		
		\medskip
		
		Hence, $P(n)$ is true for all $n$.\pause
		
		\medskip
		
		To prove that $\sum\limits_{i=n}^{m} c = (m-n+1)c$, we used the fact $P(m)$ and $P(n-1)$ is true. \pause
		
		From the first one, we have $\sum\limits_{i=1}^{m} c = \sum\limits_{i=1}^{n-1} c + \sum\limits_{i=n}^{m} c$. \pause
		
		Hence, $mc = (n-1)c + \sum\limits_{i=n}^{m} c\iff \sum\limits_{i=n}^{m} c = (m-n+1)c$. \pause
	\end{proof}
	
	To proof the next identity, one can similarly do an induction on $\sum\limits_{i=1}^{m} c\cdot f(i)$. \pause
	
	On the next slide, we shall prove the final one.
	
\end{slide}
\begin{slide}[toc=]{Properties of summation}
	\begin{proof}
		Let $P(n)$ be the proposition that 
		
		``$\sum\limits_{i=1}^{n} \left[f(i)+g(i)\right] = \left(\sum\limits_{i=1}^{n} f(i)\right)+\left(\sum\limits_{i=1}^{n}g(i)\right)$''. \pause
		
		\medskip
		
		$P(1)$ is true, as LHS is $\left[f(1)+g(1)\right]$ and the right is $\left(f(1)\right)+\left(g(1)\right)$.\pause
		
		Assume $P(k)$ is true, then 
		
		$\sum\limits_{i=1}^{k+1} \left[f(i)+g(i)\right] = \sum\limits_{i=1}^{k} \left[f(i)+g(i)\right] + \left[f(k+1)+g(k+1)\right]$\pause
		
		\medskip
		
		$\sum\limits_{i=1}^{k+1} \left[f(i)+g(i)\right] = \left(\sum\limits_{i=1}^{k} f(i)\right)+\left(\sum\limits_{i=1}^{k}g(i)\right) + \left[f(k+1)+g(k+1)\right]$\pause
		
		\medskip
		
		$\sum\limits_{i=1}^{k+1} \left[f(i)+g(i)\right] = \left(\sum\limits_{i=1}^{k} f(i)+ f(k+1)\right) + \left(\sum\limits_{i=1}^{k}g(i) + g(k+1)\right)$\pause
		
		\medskip
		
		$\sum\limits_{i=1}^{k+1} \left[f(i)+g(i)\right] = \left(\sum\limits_{i=1}^{k+1} f(i)\right) + \left(\sum\limits_{i=1}^{k+1}g(i)\right)$\pause
		
		\medskip
		
		Hence, $P(n)$ is true for all $n$.\pause
		
		Finally, the part from $n$ to $m$ can be completed by consider $1$ to $n-1$ and $1$ to $m$. \pause
		
		Note that, the rearrangement always holds if we have $f(i)+g(i)+h(i)$  and more functions.
	\end{proof}
\end{slide}
\begin{slide}[toc=Direct proof]{The direct proof for power sum of two}
	Let $f(n) = (n+1)^3 - n^3 = 3n^2 + 3n + 1$ and consider $\sum\limits_{i=1}^{n} f(i)$. \pause

	Viewing $f(n)$ as the difference of two numbers, we have \pause
	
	$\sum\limits_{i=1}^{n} \left((i+1)^3 - i^3\right) = \sum\limits_{i=1}^{n} (i+1)^3 - \sum\limits_{i=1}^{n} i^3$ \pause
	
	It is equal to $\sum\limits_{i=2}^{n+1} i^3 - \sum\limits_{i=1}^{n} i^3 = (n+1)^3 - 1$\pause
	
	On the other hand, $\sum\limits_{i=1}^{n}\left(3i^2 + 3i + 1\right)=\sum\limits_{i=1}^{n}(3i^2) + \sum\limits_{i=1}^{n}(3i) + \sum\limits_{i=1}^{n}(1)$\pause
	
	Therefore, $(n+1)^3 - 1 = \sum\limits_{i=1}^{n}(3i^2) + \sum\limits_{i=1}^{n}(3i) + \sum\limits_{i=1}^{n}(1)$\pause
	
	Rearrange the terms, we have $\sum\limits_{i=1}^{n}(3i^2) = (n+1)^3 - 1 - \sum\limits_{i=1}^{n}(3i) - \sum\limits_{i=1}^{n}(1)  $.\pause
	
	Hence, $3\sum\limits_{i=1}^{n}i^2 = (n+1)^3  - 3\sum\limits_{i=1}^{n}(i) - (n+1)$.\pause
	
	Using the identity for $1+2+\cdots +n=\frac{1}{2}n(n+1)$, we have\pause
	
	$3\sum\limits_{i=1}^{n}i^2 =  (n+1)^3 - \frac{3}{2}n(n+1) - (n+1)$.\pause
	
	factorize gives,$\frac{1}{2}(n+1)\left(2n^2 + 4n +2 - 3n -2 \right)=\frac{1}{2}(n+1)(n(2n+1))$ \pause
	
	Divide both side by $3$, we have $\sum\limits_{i=1}^{n}i^2 =\frac{1}{6}n(n+1)(2n+1)$.
\end{slide}
\begin{slide}[toc=Sum of G.S.]{Sum of geometric sequence}
	$\sum\limits_{i=0}^{n}x^i=\frac{x^{n+1}-1}{x-1}$ ,i.e. $1+x+x^2+\cdots+x^n =\frac{x^{n+1}-1}{x-1}$ \pause
	\begin{proof}
		Similarly, we let $P(n)$ be the above proposition. \pause
		
		Clearly, $P(0)$ and $P(1)$ are true.\pause
		
		For the induction step, assume $P(k)$ is true,\pause
		
		then $\sum\limits_{i=0}^{k+1}x^i = \frac{x^{k+1}-1}{x-1} + x^{k+1} = \frac{x^{k+2}-1}{x-1}$\pause
		
		Hence, by MI, the above $P(n)$ is true for all $n$.
	\end{proof}
	
	\pause
	
	Alternatively,\pause
	\begin{proof}
		$x\sum\limits_{i=0}^{n}x^i - \sum\limits_{i=0}^{n}x^i = (x-1) \sum\limits_{i=0}^{n}x^i$ \pause
		
		On the other hand, $x\sum\limits_{i=0}^{n}x^i = \sum\limits_{i=0}^{n}x^{i+1} = \sum\limits_{i=1}^{n+1}x^i$.\pause
		
		Hence, $x\sum\limits_{i=0}^{n}x^i - \sum\limits_{i=0}^{n}x^i = x^{n+1} - 1$.\pause
		
		Therefore, $x^{n+1} - 1 = (x-1) \sum\limits_{i=0}^{n}x^i$.
	\end{proof}	
	
	\pause
	
	Find $1+\frac{1}{2}+\frac{1}{4}+\cdots +\frac{1}{32768}$ using the above identity.
\end{slide}
\section{Variation of induction}
\begin{slide}{Second induction}
	The base case becomes ``$P(1)$ and $P(2)$''.
	
	The induction step becomes $P(k)\mbox{ and }P(k+1)\implies P(k+2)$.\pause
	
	Define $F_0=0,F_1=1$ and $F_n=F_{n-1}+F_{n-2} \mbox{ for }n>1$.\pause
	
	Let $r_1=\frac{1+\sqrt{5}}{2},r_2=\frac{1-\sqrt{5}}{2}$ , prove that $F_n = \frac{r_1^n+r_2^n}{\sqrt{5}}$.
	
	\begin{proof}
		Use the second induction, \pause
		
		we let the proposition $P(n)$ to be $F_{n-1} = \frac{r_1^{n-1}+r_2^{n-1}}{\sqrt{5}}$.\pause
		
		Prove that $r_1+r_2=1$ and show the induction step.\pause
		
		Left as exercise.
	\end{proof}
\end{slide}
\begin{slide}[toc=Backward induction]{Backward induction proof of the AGM-inequality}
	\begin{thm*} Let $a_1,a_2,\dots ,a_n$ be $n$ non-negative numbers.
		Then $$\frac{a_1+a_2+\cdots +a_n}{n}\geq \sqrt[n]{a_1\cdot a_2\cdots \cdot a_n }$$
	\end{thm*}
	\begin{proof}
		We proved that $P(2)$ is true and want to show that $P(2^n)$ is true.\pause
		
		Using the induction for $P(2^k)\implies P(2^{k+1})$, we have $\forall n, P(2^n)$.\pause
		
		Finally, we showed that $\forall k,P(k)\implies P(k-1)$. \pause
		
		We starts from $2\implies 4\implies 8\implies 16\implies \dots $  \pause
		
		then $4\implies 3 ,8\implies 7\implies 6\implies 5 , \dots $ \pause
		
		The detail are left as exercise.
	\end{proof}
	
	\pause
	
	Be familiar with the mathematical induction, we will use it quite often.
\end{slide}
\section{Control flows and algorithm}
\begin{slide}{Summation in computer}
	We introduce the summation notation for summing up terms. \pause
	
	However, there is no such direct concept for computer. \pause
	
	We should use the basic things that a computer can do to simulate the summation \pause

	Remember that a computer can do looping. \pause
	
	A \textbf{control flow} is a statement which tells the computer what it should perform. \pause
	
	We shall express these ideas in pseudo-code. \pause
	
	An \textbf{algorithm} is a sequence of operations to tell the computer how to solve a problem. \pause
\end{slide}
\begin{slide}[toc=Pseudo-code]{Pseudo-code - Assignment}
	Pseudo-code is a hand-written notation to specify how should a computer to perform a task. \pause
	
	The code is merely for human being. \pause
	
	It represents what the computer can do. \pause
	
	In pseudo-code, we denote ``$x\gets 1$'' to represent storing $1$ to $x$. \pause
	
	This is called an \textbf{assignment}. \pause
	
	In general, the grammar structure for an assignment expression is 
	
	<$Assignment$>$::=$<$Name$> $\gets$ <$Value$>\pause
\end{slide}
\begin{slide}[toc=]{IF-THEN / IF-THEN-ELSE}
	IF-THEN and IF-THEN-ELSE are control flow which have two and three parts respectively.\pause
	
	They tells the computer to perform task based on a given condition. \pause
	
	\begin{enumerate}
		\item	The condition for checking.\pause
		\item	The actions to be performed once the condition holds.\pause
		\item 	The actions to be performed once the conditions is false. (Only for IF-THEN-ELSE) \pause
	\end{enumerate}		
	
	The only difference between them is that, \pause
	
	when the condition does not holds, the IF-THEN-ELSE tells the computer to do some task. \pause
	
	This is called a \textbf{conditional flow}. \pause

	Usually, we denote it by ``\textbf{if} \textit{condition} \textbf{then} \textit{action list}''. \pause
	
	or ``\textbf{if} \textit{condition} \textbf{then} \textit{action list} \textbf{else} \textit{action list}''. \pause
	
	In general, the grammar structure for a conditional flow is \pause
	
	<\textit{IF-THEN}>$::=$\textbf{if} <\textit{Condition}> \textbf{then} <\textit{Actions-for-true}> \pause
	
	<\textit{IF-THEN-ELSE}>$::=$\textbf{if} <\textit{Condition}> \textbf{then} <\textit{Actions-for-true}> \textbf{else} <\textit{Actions-for-false}> \pause	
\end{slide}
\begin{slide}[toc=]{Example - Condition flow}
	You are given a positive integer $x$, we want to find the next even number.\pause
	
	How can we tell the computer to find the next even number?\pause
	
	The following is an algorithm. \pause

	\medskip
	
	\begin{algorithm}[H]
		\SetAlgoLined
		\KwData{$x$ is a positive integer.}
		\KwResult{The even number just after $x$.}		
		\eIf{$x$ is odd}{
			\Return{$x+1$}
		}{
			\Return{$x+2$}
		}
	\end{algorithm}
	
	\pause
	
	How can we check if $x$ is odd?\pause
	
	\medskip
	
	\begin{algorithm}[H]
		\SetAlgoLined
		\KwData{$x$ is a positive integer.}
		\KwResult{The even number just after $x$.}		
		\eIf{$x\bmod 2\neq 0$}{
			\Return{$x+1$}
		}{
			\Return{$x+2$}
		}
	\end{algorithm}
\end{slide}
\begin{slide}[toc=]{Looping - Iterative}
	In the summation notation, we start from $i=n$ and then count one by one until $i=m$. \pause
	
	A similar control flow called the FOR-LOOP also exists in computer.\pause
	
	It consists of three parts, \pause
	
	\begin{enumerate}
		\item	The starting/initial value $n$ \pause
		\item	The ending value $m$ \pause		
		\item	Actions to be performed. \pause		
	\end{enumerate}
	
	It is usually called a \textbf{for-loop} or an \textbf{iterative loop}.\pause
	
	The grammar is <FOR-LOOP>$::=$\textbf{for} <name> $\gets$ <value> \textbf{to} <value> \textbf{do} <Actions>.\pause
	
	Therefore, the summation notation $\sum\limits_{i=1}^{10}i^2$ can be simulated as,\pause
	
	\medskip

	\begin{algorithm}[H]
		\SetAlgoLined
		$t\gets 0$ \;
		\For{$i\gets 1$ \KwTo $10$}{
			$t\gets t+i^2$
		}
		\Return{$t$}
	\end{algorithm}
\end{slide}

\begin{slide}[toc=]{Looping - Conditional}
	Usually, we would like to repeat a task whenever the condition holds.\pause
	
	For example, in checking password, 
	
	the computer keeps asking for a password until a correct one is entered.\pause
	
	A similar control flow called the WHILE-LOOP also exists in computer.\pause
	
	It consists of two parts, \pause
	
	\begin{enumerate}
		\item	The condition $n$ \pause
		\item	Actions to be performed. \pause		
	\end{enumerate}
	
	It is usually called a \textbf{while-loop} or a \textbf{conditional loop}.\pause
	
	The grammar is <WHILE-LOOP>$::=$\textbf{while} <condition> \textbf{do} <Actions>.\pause
	
	Therefore, a password checking algorithm is as follow,\pause
	
	\medskip
	
	\begin{algorithm}[H]
		\SetAlgoLined
		\KwIn{Password}
		\While{Password is wrong}{
			\KwOut{You enter a wrong password.}
			\KwIn{Password.}
		}
		\KwOut{Login successfully.}
	\end{algorithm}
\end{slide}
\section{C's Syntax}
\begin{slide}{Character and String}
	Previous lecture, we know that a computer can recognize value.\pause
	
	A computer integer is of the form [SIGN]?[0-9]+.\pause
	
	A computer real number is given similarly.\pause
	
	These two are called \textbf{numerical values}.\pause
	
	Another type of values the computer can recognize is ``English''. \pause
	
	It can be divided into two types. \pause
	
	A ``letter'' or a ``vocabulary''. \pause
	
	The computer-letter is called a \textbf{character}.\pause
	
	The computer-vocabulary is called a \textbf{string}.\pause
	
	Unlike English, we do not impose restriction on whatever we called a string.\pause
	
	There is no spell checking.\pause
	
	A character in C is enclosed by two single quotation mark. \pause
	
	For example, 'a' , 'b' , ' '. \pause
	
	A string in C is enclosed by two double quotation mark.\pause
	
	For example, "abc" , "hot". \pause
	
	As similar to the regular expression, whenever a character has special meaning, use a \textbackslash.\pause
	
	Therefore, to denote a character single quote, we should use '\textbackslash''.\pause
	
	And similar, a string with a double quote, we should use "asd\textbackslash"".\pause
	
\end{slide}

\section{End}
\end{document}
