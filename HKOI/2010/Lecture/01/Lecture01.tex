%% LyX 1.6.5 created this file.  For more info, see http://www.lyx.org/.
%% Do not edit unless you really know what you are doing.
\documentclass[size=8pt,english,style=fyma,display=slidesnotes,mode=present,paper=screen]{powerdot}
\usepackage[T1]{fontenc}
\usepackage[utf8]{inputenc}
\usepackage{setspace}
\usepackage{amsthm}
\usepackage{amsmath}
\usepackage{tikz}
\usepackage{graphics}
\onehalfspacing

\makeatletter

%%%%%%%%%%%%%%%%%%%%%%%%%%%%%% LyX specific LaTeX commands.
\pdfpageheight\paperheight
\pdfpagewidth\paperwidth

	%%%%%%%%%%%%%%%%%%%%%%%%%%%%%% Textclass specific LaTeX commands.
	\theoremstyle{plain}
	\newtheorem*{thm*}{Theorem}
	\newtheorem{thm}{Theorem}	
	\newtheorem*{def*}{Definition}
\makeatother

\usepackage{babel}

\begin{document}
	
\title{HKOI Training}
\author{$ami\sim wkc$}
\date{Last modified: \today}
\maketitle
\lyxend\section[toc=Introduction]{Lecture 01 \\ An introduction to problem solving and programming in C}

\begin{slide}{Problem solving - Maths}
	\begin{itemize}
		\item	Understand the problem \pause
		\item	Discover new properties, lemmas, theorems etc.	\pause
		\item	Understanding what it is \pause
		\item	Understanding how to use \pause
		\item	Understanding why it is true \pause
	\end{itemize}	

\end{slide}
\begin{slide}[toc=Pythagoras theorem]{Example - Pythagoras theorem}
	\input{pic/pyth.tkz}
	
	Find $BC$ if $BC$ is an integer.
	
	Find positive integer solutions to the equation $a^2+b^2=c^2$.
	
	Is there any relation between the above two problems?	
\end{slide}
\begin{slide}[toc=]{}
	\begin{thm*}[Pythagoras]
		If $a^2+b^2=c^2$, there is a right triangle with sides $a$,$b$,$c$.
		
		For each right triangle, sum of square of legs is equal to square of the hypotenuse.		
	\end{thm*}
	In particular, if a right triangle has integer sides, the sides are a solution to the equation.
\end{slide}
\begin{slide}[toc=]{Example - Pythagoras theorem (Cont'd)}
	What is the Pythagoras theorem?\pause
		\begin{itemize}
			\item It relates the sides of a right triangle.
			\item It relates each solution with a right triangle.
		\end{itemize}\pause
	How can you use it to find $BC$?\pause
	
	The theorem said, $\left(3,4,BC\right)$ is one of the solutions.
	
	There are a few possibilities:\pause
		\begin{enumerate}
			\item $b=3, c=4$ and $a = BC$
			\item $b=4, c=3$ and $a = BC$
			\item $a=3, c=4$ and $b = BC$
			\item $a=4, c=3$ and $b = BC$
			\item $a=3, b=4$ and $c = BC$
			\item $a=4, b=3$ and $c = BC$
		\end{enumerate}\pause
		
	\textbf{Are they all possible? } \pause  NO!
\end{slide}
\begin{slide}[toc=]{Example - Pythagoras theorem (Cont'd)}
	Let's check all of them:
	\begin{enumerate}
		\item putting $b=3, c=4$ and $a = BC$, we have $BC^2+3^2=4^2\iff BC^2=16-9=7 \implies BC=\sqrt{7}$ (?)
		\item putting $b=4, c=3$ and $a = BC$, we have $BC^2=-7$, impossible
	\end{enumerate}
	\vdots
	
	Finally, putting $a=4, b=3$ and $c = BC$, we have $BC^2=25 \implies BC=5$.
	
	We have the two possible values for $BC$, $\sqrt{7}$ and $5$.
	
	Since $\sqrt{7}$ is not an integer, $BC$ must be $5$.\footnote{
		\begin{small}``When you eliminate the impossible, whatever remains--however improbable--must be the truth.''\end{small}
	}
\end{slide}	
\begin{slide}[toc=]{Example - Pythagoras theorem (Cont'd)}
	\textbf{Why is the Pythgoras theorem true?}\pause

	\input{pic/pyth_proof.tkz}
\end{slide}
\begin{slide}[toc=]{Example - Pythagoras theorem (Cont'd)}
	\begin{proof}
		Rotate the given right triangle to produce the figure.
		
		Area of the whole figure is $\left(a+b\right)^2=a^2+2ab+b^2$
		
		On the other hand, it is the sum of area the smaller square and four triangles.
		
		Area of $\triangle ABC$ is $\frac{1}{2}ab$, so does the other three triangles.
		
		Area of the smaller square is $c^2$
		
		Hence, $a^2+2ab+b^2 = 4\left(\frac{1}{2}ab\right) + c^2 \iff a^2+b^2=c^2$.	
	\end{proof}
\end{slide}		
\begin{slide}{Remainder/Modulus}
	\begin{def*}
		Let $m,n$ be two integers with $n\neq 0$,
		
		$m \bmod n$ is defined as the remainder when $m$ is divided by $n$ 
	\end{def*}
	
	For example, $7 \bmod 3 = 1$ and $107 \bmod 8 = 3$\pause
	\medskip
	
	Let $n = 7\underbrace{201120112011\dots 20112011}_{2000-digits}$ be an $2001$-digit number.
	
	\medskip{}
	Find $n \bmod 3$ and $n \bmod 11$. \pause (Joking!?) \pause
	
	Hints: Consider digit sum and alternating digits sum. \pause	

	\begin{thm*}[Divisibility of 3 and 11]
		Let $m$ be an integer, 
		
		$S$ be its digit sum and $A$ be its alternating digit sum,
		
		then $m \bmod 3 = S \bmod 3$ and $m \bmod 11 = A \bmod 11$
	\end{thm*}\pause

	As a demonstration,\pause
	
	Alternating digit sum of 10947 : 7 - 4 + 9 - 0 + 1 = 13    (adding from the rightmost digit)
	
	$10947 = 11\cdot 995 + 2$ hence $10947 \bmod 11 = 2$. Also, $13\bmod 11=2$.\pause
	
	Digit sum of 1234 : 1 + 2 + 3 + 4 = 10   (Easy)
	
	$1234=3\cdot 411 + 1$ hence $1234\bmod 3 = 1$. Also $10\bmod 3=1$.\pause
	
	We will come back to the proof when we have enough maths knowledge. (Number theory)
\end{slide}
\begin{slide}{Problem solving - Computer}
	An extra tool - computer. \pause
	
	It can do task very fast. \pause
	
	This is the only advantage we got from it. \pause
	
	NO ANY MORE. \pause
	
	It is a rubbish, it cannot understand our languages. \pause
	
	It is more or less the same as your calculator. \pause
\end{slide}
\begin{slide}[toc=PCIMC-09-H5-Q16]{Guess number}
	$n$ is a five-digit square number, whose digits are $2$ and $9$ only. Find all possible $n$.\pause
	
	Computer aided strategy:\pause
	\begin{enumerate}
		\item List out all the five-digit square numbers
		\item Check the numbers one by one to see if its digits are $2$ and $9$ only.
	\end{enumerate}\pause
		
	Mathematical way: (only idea are listed here)\pause
	
	\begin{small}
		Let $n=\overline{abc}^2$ and $n=\overline{ABCDE}$
		\begin{enumerate}
			\item The unit digit must be $9$ and $c=3,7$ \pause  (Why?) \pause
			\item The tenth's digit must be $2$ \pause (Key step!) \pause
			\item $b=2,7$ \pause  (Why?) \pause
			\item Since $323^2 > 320^2 = 102400$, we have $a = 1, 2$ \pause
			\item If $a=2$ then  $ 2 < A < 9 $ , impossible \pause
			\item $a=1$ and guess $123^2$ , $173^2$ , $127^2$ and $177^2$. \pause
			\item $n=29929=173^2$
		\end{enumerate}
	\end{small}
	
	We will come back to step 2 and 3 when we have enough maths knowledge. (Number theory)
\end{slide}
\begin{slide}[toc=]{Guess number (Cont'd)}
	Does the following computer related idea work?
	
	\begin{itemize}
		\item Ask the computer to solve it.
		\item Ask the computer to explain the previous questions.
	\end{itemize}\pause
	
	NO, computer can never beat human being. \pause
	
	It is a deep result in mathematical logic that, 
	
	given a list of assumptions, there are something true but one cannot prove it.\pause
	
	Computer can only follow a list of instruction and perform it.\pause
	
	Prime checking example, to check whether 1001 is a prime.
	\begin{enumerate}
		\item Set $z = 1$ at first,
		\item Starting from 2 to 1000 : if a number divide 1001, change $z$ to $0$.
		\item If $z$ is $1$ then 1001 is a prime, otherwise it is a composite.
	\end{enumerate}
	
	Computer can follow the above instruction and do each step one by one repeatedly.
\end{slide}
\section{Programming in C}
\begin{slide}{What can a computer do?}
	\begin{itemize}
		\item	Arithmetic - Addition, subtraction, multiplication, quotient, modulus and division \pause
		\item	Store value - Putting a value into some named boxes \pause
		\item	Change value - Change a value to another value in a given boxes \pause		
		\item	Store action - Record all steps for computing $\sqrt{x}$ etc. \pause
		\item 	Perform action - to perform any defined action like finding $\sqrt{29929}$ \pause
		\item	Comparison - equal, not equal, greater than, greater than or equal to ... \pause
		\item	Logical comparison -  NOT , AND , OR , XOR (exclusive OR) \pause 
		\item	Conditions - do different actions based on a condition \pause
		\item	Looping - Repeated a list of actions \pause
		\item	Recognize a value - The integer 1001 , $\pi$ , ``Your Name here'' etc. \pause
	\end{itemize}
	
	For those who interested finding square root without calculator, read \href{http://en.wikipedia.org/wiki/Methods_of_computing_square_roots\#Digit-by-digit_calculation}{wiki page}
\end{slide}
\begin{slide}{Linguistic matter of human language}
	Although we list out the steps for checking 1001 is a prime or not in English,
	
	a computer cannot understand our human language. 
	
	Our language has so many grammar rules, even ourselves would feel confusing sometimes.\pause
	
	Consider the followings:\pause
	\begin{enumerate}
		\item He looks so blue.
		\item The second unique child of the God.
	\end{enumerate}\pause
	
	The first sentence has two meaning according to different meaning of blue.
	
	The second sentence is self-contradicting.\pause
	
	Computer cannot distinguish the meanings in these situations.
\end{slide}
\begin{slide}{Programming Language - Computer's language}
	Therefore, we need a simpler language that can describe the things for a computer can do.\pause
	
	These kind of languages is called a \textit{programming language}.\pause
	
	\medskip{}
	
	Programming Languages:\pause
	
	C, C++, Pascal, Common Lisp, Go, Prolog, ML, PHP, MATHLAB, Assembly, Machine Code, ...\pause

	\medskip{}
	
	These languages are just like Chinese, English, Japanese, Spanish, French, ...\pause
	
	Programming languages has its grammar rules, punctuation marks, sentence structure etc...\pause
	
	First of all, spacings and lines are not important to computer. (it doesn't read by eyes) \pause
	
	\medskip{}
	
	Even worst, if you did write something contain grammatical error or missing punctuation etc.,
	
	a computer can never correct your mistake and it will just merely complain.
	
	In the training course, we will use C programming language and C++ later.\pause
	
	\textit{Syntax} is the term for Programming Languages' grammar rules and punctuation marks.
\end{slide}
\begin{slide}{C Syntax - Arithmetic}
	\twocolumn[]{
		Addition	
		
		Subtraction
		
		Multiplication
		
		Quotient
		
		Modulus
		
		Open Bracket
		
		Close Bracket
		
	}{
		+
		
		-
		
		*
		
		/
		
		\%
		
		(
		
		)
	}
	
	\medskip{}
	
	Unlike mathematics, the multiplication symbol is a star * in C.
	
	Therefore,  the expression $1+2\times 3$ is written as $1+2*3$ in C. \pause
	
	It has no power index in C, so $x^3$ must be written as $x*x*x$ in C. \pause
	
	Obviously, you cannot type fraction easily in a computer.
	
	$\frac{3+x}{a-2}$ must be written as $(3+x)/(a-2)$ in C. \pause
	
	\medskip{}
	
	\emph{WARNING}, $(3+x)/(a-2)$ and $3+x/a-2$ are not the same.
	
	$3+x/a-2$ means $3+\frac{x}{a}-2$. \pause

	\medskip{}
	
	However many spaces are there, $3+x/a-2$  and $3+x\,\,\, /\,\,\,\,  a-2$ are the same.
	
	Therefore, we need to be strict in the rules that first multiplication then addition.
	
\end{slide}
\section{End}
\end{document}
