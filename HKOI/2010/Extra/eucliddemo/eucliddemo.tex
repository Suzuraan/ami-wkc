%% LyX 1.6.5 created this file.  For more info, see http://www.lyx.org/.
%% Do not edit unless you really know what you are doing.
\documentclass[size=8pt,english,style=fyma,display=slidesnotes,mode=present,paper=screen]{powerdot}
\usepackage[T1]{fontenc}
\usepackage[utf8]{inputenc}
\usepackage{setspace}
\usepackage{amsthm}
\usepackage{amsmath}
\usepackage{tikz}
\usepackage{graphics}
\usepackage{calc}
\usepackage{forloop}
\usepackage{ifthen}

\newcounter{flag}
\newcounter{EuclidA} 
\newcounter{EuclidB} 
\newcounter{EuclidR} 
\newcounter{EuclidQ} 
\newcounter{EuclidXA}
\newcounter{EuclidXB}
\newcounter{EuclidYA}
\newcounter{EuclidYB}
\newcounter{EuclidXTemp}
\newcounter{EuclidYTemp}


\newcommand{\doEuclideanAlgorithm}[2]{

\setcounter{EuclidA}{#1}
\setcounter{EuclidB}{#2}
\setcounter{EuclidXA}{1}
\setcounter{EuclidYA}{0}
\setcounter{EuclidXB}{0}
\setcounter{EuclidYB}{1}
%\forLoop{1}{4}{identRow} { \forLoop{1}{4}{identCol} { \ifthenelse{ \equal{ \value{identRow} }{ \value{identCol} } }{ 1 }{ 0 } \ifthenelse{ \equal{\value{identCol}}{10} }{}{ & } } \\ }

%\arabic{EuclidR}	&	\arabic{EuclidQ}	&	\arabic{EuclidXB}	&	\arabic{EuclidYB}	\\ \hline
%\arabic{EuclidR}	\;	\arabic{EuclidQ}	\;	\arabic{EuclidXB}	\;	\arabic{EuclidYB}	\\
	
$$\begin{array}[t]{|c||c|c|c|c|c|} \hline
	k	&	r_k					&	q_k	&	x_k	&	y_k	&	\mbox{Formulae}	\\ \hline \hline
	0	&	\arabic{EuclidA}	&		&	1	&	0	&		\\ \hline
	1	&	\arabic{EuclidB}	&		&	0	&	1	&		\\ \hline	
	\forloop[1]{flag}{2}{\value{EuclidB} > 0}{
		\setcounter{EuclidQ}{\value{EuclidA} / \value{EuclidB}} 
		\setcounter{EuclidR}{\value{EuclidA} - \value{EuclidQ} * \value{EuclidB}}
		\setcounter{EuclidXTemp}{\value{EuclidXA} - \value{EuclidXB} * \value{EuclidQ}}
		\setcounter{EuclidYTemp}{\value{EuclidYA} - \value{EuclidYB} * \value{EuclidQ}}
		\arabic{flag}	&	\arabic{EuclidR}	&	\arabic{EuclidQ}	&	\arabic{EuclidXTemp}	&	\arabic{EuclidYTemp}	&	\arabic{EuclidA}=\arabic{EuclidB}\cdot\arabic{EuclidQ}+\arabic{EuclidR}
		\setcounter{EuclidA}{\value{EuclidB}}
		\setcounter{EuclidB}{\value{EuclidR}}		
		\setcounter{EuclidXA}{\value{EuclidXB}}
		\setcounter{EuclidXB}{\value{EuclidXTemp}}		
		\setcounter{EuclidYA}{\value{EuclidYB}}
		\setcounter{EuclidYB}{\value{EuclidYTemp}}				
		\ifthenelse{\value{EuclidB} > 0}{\\\hline}{}
	}	\\\hline
	
\end{array}$$
}



\onehalfspacing

\makeatletter

%%%%%%%%%%%%%%%%%%%%%%%%%%%%%% LyX specific LaTeX commands.
\pdfpageheight\paperheight
\pdfpagewidth\paperwidth

%%%%%%%%%%%%%%%%%%%%%%%%%%%%%% Textclass specific LaTeX commands.
\theoremstyle{plain}
\newtheorem*{thm*}{Theorem}
\newtheorem*{example}{Example}
\newtheorem*{rem*}{Remark}
\makeatother

\usepackage{babel}

\begin{document}
	
	\title{HKOI Training}
	\author{$ami\sim wkc$}
	\date{Last modified: \today}
	\maketitle
	\lyxend\section{Euclidean algorithm}
	
\begin{slide}{Finding integer solutions}
	Finding integers solutions to $123 x+9y=3$.	\pause
	
	\begin{rem*}
		Here, all the unknowns are assumed to be integers.
	\end{rem*}

	\pause

	\begin{onehalfspace}
		\doEuclideanAlgorithm{123}{9}
	\end{onehalfspace}
	
	To understand the euclidean algorithm for finding a parituclar solution.	\pause
	
	First, we have $123=9\left(13\right)+6\iff123\left(1\right)+9\left(-13\right)=6$	\pause
	
	From the second formulae, $9=6\left(1\right)+3\iff9\left(1\right)-6\left(1\right)=3$	\pause
	
	i.e. $9\left(1\right)-\left(123\left(1\right)+9\left(-13\right)\right)\left(1\right)=3\iff123\left(-1\right)+9\left(14\right)=3$	\pause
	
	Through this, you should know the new $x_{k}$ and $y_{k}$ are obtained
	
	by $x_{k-2}-q_{k}x_{k-1}$ and $y_{k-2}-q_{k}y_{k-1}$.	\pause
	\begin{rem*}
		Remember that, we always read the last but one row to find a solution.
	\end{rem*}
\end{slide}
\begin{slide}{Solving modulus equation}
	Find all integers $x$ such that $7x\bmod13=1$.	\pause
	
	\begin{proof}
		The remainder is $1$ means that $7x$ is the sum a multiple of $13$
		and $1$.	\pause
		
		Mathematically, this is equivalent to $7x=13m+1$ for some $x$ and
		$m$.	\pause
		
		Rewrite it into $7x-13m=1$ and let $y=-m$.	\pause
		
		It becomes $7x+13y=1$ and should have infinitely many solutions.	\pause
		
		Finally, using the euclidean algorithm, 	\pause
		
		\begin{onehalfspace}
			\doEuclideanAlgorithm{13}{7}
		\end{onehalfspace}
		
		we find $x_{0}=2$.	\pause
		
		\begin{doublespace}
			Hence, all the integers $x$ are of the form $2+13k$ where $k$ is
			any integers.\end{doublespace}
			
	\end{proof}
\end{slide}
\section{End}
\end{document}
